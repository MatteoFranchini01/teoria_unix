\documentclass[italian,12pt,a4paper]{article}
\usepackage[utf8]{inputenc}
\usepackage[T1]{fontenc}
\usepackage{listings}
\usepackage{babel}
\usepackage{hyperref}
\usepackage{graphicx}
\usepackage{tocloft}
\title{Teoria UNIX}
\author{Matteo Franchini}

\renewcommand{\cftfigpresnum}{Figura~}
\renewcommand{\cftfigaftersnum}{:}
\setlength{\cftfigindent}{0pt}
\setlength{\cftfignumwidth}{5em}
\renewcommand{\listfigurename}{Elenco delle Figure}

\begin{document}
	\maketitle
	\tableofcontents
	\section{Programmazione di sistema UNIX}
	\subsection{Argomenti di programma}
	Un programma può accedere agli eventuali argomenti di invocazione attraverso i parametri della funzione principale \textbf{main}
	\begin{lstlisting}[xleftmargin=-5.5cm]
		main (int argc, char *argv[]) {
			int i;
			printf("Numero di argomenti = %d\n", argc);
			for (i = 0; i < argc; i++) {
				printf("Argomento %d (argv[%d]) = %s\n", i, i, argv[i]);
			}
		}
	\end{lstlisting}
	si noti che \verb|%d| si usa per indicare che in quel punto ci va un \textbf{intero}, mentre \verb|%s| si usa per indicare che ci va una \textbf{stringa}
	\subsection{Compilazione}
	\begin{verbatim}
		gcc -o mioprogramma mioprogramma.c
	\end{verbatim}
	\subsection{Eseguendo il programma}
	\begin{verbatim}
		./mioprogramma 1 pippo pluto 4
	\end{verbatim}
	\newpage
	\subsection{Variabile di ambiente}
	\begin{lstlisting}[xleftmargin=-1 cm]
main(int argc, char *argv[], char **envp) {
	int i;
	prinft("Numero di argomenti (argc) = %d\n", argc);
	for (i = 0; i < argc; i++) {
		printf("Argomento %d (argv[%d]) = %s\n", i, i, argv[i]);
	}
	while (*envp != NULL) { printf("%s\n", envp++); }
}
	\end{lstlisting}
	\subsection{Perror e Streerror}
	\verb|perror| e \verb|strerror| permettono di visualizzare o di generare messaggi descritti dell'errore
	\begin{lstlisting}
if (syscall_N (..., ...) < 0)
{
	perror("Errore nella syscall_N");
	/*
	la descrizione dell'errore viene concatenata 
	alla stringa argomento
	*/
	exit(1); // terminazione del processo con errore	
}
	\end{lstlisting}
\section{Operazioni sui file}
\subsection{Apertura file}
\textbf{Apertura ed eventuale creazione di un file}
\begin{lstlisting}
#include <sys/types.h>
#include <sys/stat.h>
#include <fcntl.h>

int open (const char *pathname, int flags);
\end{lstlisting}
\textbf{oppure}
\begin{lstlisting}
#include <sys/types.h>
#include <sys/stat.h>
#include <fcntl.h>

fd = int open (const char *pathname, int flags, mode_t mode);
\end{lstlisting}
\begin{itemize}
	\item \verb|pathname| è il nome del percorso
	\item \verb|flags| contiene il modo di accesso richiesto: uno tra \verb|O_RDONLY|, \verb|O_WRONLY|, \verb|O_RDWR| più altre eventuali OR\\
	Esempio: 
	\begin{verbatim}
		O_WRONLY|O_CREAT|O_TRUNC
	\end{verbatim}
	per richiedere la creazione di un nuovo file o per azzerarlo se già esiste
	\item \verb|mode| indica i \textbf{diritti di accesso}
\end{itemize}
Se l'\textbf{invocazione della primitiva} \verb|open| ha successo, viene restituito al processo un valore intero $\geq 0$ che costituisce il \textbf{file descriptor (fd)} per quel file
\subsection{Duplicazione file descriptor}
\begin{lstlisting}
#include <unistd.h>

int dup (int oldfd);
int dup2(int oldfd, int newfd);
\end{lstlisting}
\subsection{Chiusura di un file descriptor}
\begin{lstlisting}
#include <unistd.h>

int close (int fd);
\end{lstlisting}
\subsection{Lettura e scrittura di un file descriptor}
\begin{lstlisting}
#include <unistd.h>

int read (int fd, void *buf, size_t count);
int write (int fd, void *buf, size_t count);
\end{lstlisting}
\begin{itemize}
	\item \verb|read| prova a leggere dall'oggetto a cui si riferisce \verb|fd| fino a \verb|count| byte, memorizzandoli a partire dalla locazione \verb|buf|
	\item \verb|write| prova a scrivere sull'oggetto a cui si riferisce \verb|fd| fino a \verb|count| byte, letti a partire dalla locazione \verb|buf||
\end{itemize}
\subsection{Esempio di lettura/scrittura}
\begin{lstlisting}[xleftmargin=-1.5cm]
#include <sys/types.h>
#include <sys/stat.h>
#include <fcntl.h>
#include <stdlib.h>
#include <unistd.h>

#define BUFSIZ 4096

main () {
	char *f1 = "filesorg";
	char *f2 = "/tmp/filedest";
	char buffer[BUFSIZ];
	int infile, outfile; // file descriptor
	int nread;
	
	// apertura file sorgente
	
	if ((infile = open(f1, O_RONLY)) < 0) {
		perror("Apertura f1");
		exit(1);
	}
	
	/* open mi permette di aprire il file
	 * quando scriviamo infile = ... stiamo assegnando
	 * il valore restituito dalla chiamata open a infile
	 * in modo da poter accedere al file utilizzando
	 * direttamente infile, invece che il file descriptor
	 * metto la condizione < 0 in quanto se abbiamo un 
	 * errore nell'apertura il fd < 0
	 */
	 
	 // creazione file destinazione
	 if ((outfile = open(f2, WRONLY|O_CREAT|O_TRUNC, 0644)) < 0) {
	 	perror("Creazione f2");
	 	exit(2);
	 }
	 
	 /* ho messo O_WRONLY or O_CREAT or O_TRUNC
	  * in quanto se esiste il file ci scrivo sopra, 
	  * altrimenti lo creo
	  * oppure lo sovrascrivo
	  */
	
	// ciclo di lettura/scrittura 
	
	while ((nread = read(infile, buffer, BUFSIZ)) > 0) {
		if (write (outfile, buffer, nread) != read) {
			perror("Errore write");
			exit(3);
		}
		if (nread < 0) {
			perror("Errore read");
			exit(4);
		}
	}
	close(infile);
	close(outfile);
	exit(0);
}
\end{lstlisting}
\subsection{Trasferire dati tra descrittori}
\begin{lstlisting}[xleftmargin=-1cm]
#include <sys/sendfile.h>

ssize_t sendfile (int out_fd, int in_fd, off_t *offset, size_t count);
\end{lstlisting}
\verb|sendfile| copia dati da un file descriptor all'altro \textbf{rimanendo all'interno del kernel}, quindi è più efficiente dell'uso combinato di read e write che trasferiscono dati tra spazio utente e kernel
\begin{itemize}
	\item Se \verb|offset| non è NULL, indica l'indirizzo di una variabile contenente lo spiazzamento da cui iniziare la lettura da \verb|in_fd| che sarà modificata all'offset successivo all'ultimo byte letto; \verb|count| è il numero di byte da copia
	\item Se \verb|offset| non è NULL, allora \verb|sendfile()| non modifica il file offset di \verb|in_fd| altrimenti esso viene aggiustato per riflettere il numero di byte letti da \verb|in_fd|
\end{itemize}
\subsection{Copia di file con sendfile}
\begin{lstlisting}[xleftmargin=-1.5 cm]
#include <fcntl.h>
#include <stdlib.h>
#include <stdio.h>
#include <sys/sendfile.h>
#include <sys/stat.h>
#include <sys/types.h>
#include <unistd.h>

int main (int argc, char* argv[]) {
	int read_fd, write_fd;
	struct stat stat_buf;
	off_t offset = 0;
	
	// apertura file input
	
	read_fd = open(argv[1], O_RDONLY);
	
	/* con fstat otteniamo le informazioni
	 * del file che viene aperto e le salviamo
	 * all'interno di stat_buf
	 * in particolare in questo caso lo facciamo
	 * per ottenere la dimensione del file
	 */
	
	fstat(read_fd, &stat_buf);
	
	/* apriamo il file di lettura con gli stessi
	 * permessi del file sorgente "stat_buf.st_mode"
	 */
	 
	 write_fd = open (argv[2], O_WRONLY | O_CREAT, stat_buf.st_mode);
	 
	 sendfile (write_fd, read_fd, &offset, stat_buf.st_size);
	 
	 close (read_fd);
	 close (write_fd);
	 
	 return 0;
}
\end{lstlisting}
\subsection{Informazioni su file (ordinari, speciali, direttori)}
\begin{lstlisting}
#include <sys/stat.h>
#include <unistd.h>

int stat (const char *filename, struct stat *buf);
int fstat (int fd, struct stat *buf);	
\end{lstlisting}
\begin{figure}[h!]
	\centering
	\includegraphics[width=1 \linewidth]{/Users/matteofranchini/Library/CloudStorage/OneDrive-UniversitàdegliStudidiParma/Università/Appunti e slide/II anno/Screen slide/Screenshot 2023-09-02 alle 15.54.32.png}
	\caption{Struttura buf}
\end{figure}
\subsection{Cancellazione di file}
\begin{lstlisting}
#include <unistd.h>

int unlink (const char *filename);
\end{lstlisting}
Il file viene cancellato solo se: si tratta dell'ultimo link al file, non vi sono altri processi che lo hanno aperto
\section{Primitive per la gestione degli accetti}
\subsection{Creazione di un nuovo processo}
\begin{lstlisting}
#include <unistd.h>

int fork (void);
\end{lstlisting}
Viene creato un nuovo processo (figlio) identico al processo padre che ha invocato \verb|fork()|.\\
\textbf{Solo il valore di uscita della} \verb|fork| \textbf{è diverso per i due processi}
\begin{verbatim}
	pid = fork();
\end{verbatim}
per il padre \verb|pid| vale il pid del figlio, per il figlio \verb|pid = 0| 
\subsection{Sistema di generazione}
\begin{lstlisting}
#include <unistd.h>

if (fork() = = 0) {
	// codice del FIGLIO
}
else {
	// codice del PADRE
}
\end{lstlisting}
il padre può decidere se continuare la propria esecuzione concorrentemente a quella del figlio, oppure attendere che il figlio termini (\textbf{primitiva } \verb|wait|)
\subsection{Identificazione dei processi}
\begin{lstlisting}
#include <unistd.h>

pid_t getpid (void);
pid_t getppid (void);
\end{lstlisting}
La \verb|getpid| ritorna al processo chiamante il suo PID, mentre \verb|getppid| ritorna al processo chiamante l'identificatore di processo di suo padre (PID del PADRE)
\subsection{Sincronizzazione tra padre e figlio}
\begin{lstlisting}
#include <sys/types.h>
#include <sys/wait.h>

pid_t wait (int *status);
\end{lstlisting}
Il processo chiamante rimane bloccato in attesa della terminazione di uno tra i suoi figli.\\
Se la \verb|wait| ha successo il valore di ritorno è il PID del processo figlio che è terminato
\subsection{Uso della wait}
\begin{lstlisting}
int status;

if (fork () = = 0) {
	// Codice del figlio
}
else {
	// Codice del padre
	...
	// Attende che il figlio termini
	wait(&status);
}
\end{lstlisting}
Nel caso di più figli in esecuzione può essere necessario attenere \textbf{la terminazione di uno specifico figlio}
\begin{verbatim}
	while (pid = wait(&status) != pidfiglio);
\end{verbatim}
oppure direttamente
\begin{verbatim}
	waitpid(pidfiglio, &status, NULL)
\end{verbatim}
\subsection{Terminazione volontaria di un processo}
\begin{verbatim}
	#include <stdlib.h>
	void exit(int status);
\end{verbatim}
\begin{verbatim}
	#include <unistd.h>
	void _exit(int status)
\end{verbatim}
Un processo \textbf{termina volontariamente} invocando la primitiva \verb|_exit| oppure la funzione \verb|exit| della libreria standard I/O di C
\subsection{Esecuzione di un programma}
\begin{verbatim}
#include <unistd.h>

int execve (const char *pathname, char *const argv[], char *const envp[]);
\end{verbatim}
Il processo chiamante passa ad eseguire il programma \verb|filename|
La fork \textbf{crea un nuovo processo identico al padre}, la exec permette \textbf{di modificare l'ambiente di esecuzione di un processo}
\subsection{Esempio di uso della execve}



\newpage
\listoffigures
\end{document}